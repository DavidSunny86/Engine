%%%%%%%%%%%%%%%%%%%%%%%%%%%%%%%%%%%%%%%%%
% Thin Sectioned Essay
% LaTeX Template
% Version 1.0 (3/8/13)
%
% This template has been downloaded from:
% http://www.LaTeXTemplates.com
%
% Original Author:
% Nicolas Diaz (nsdiaz@uc.cl) with extensive modifications by:
% Vel (vel@latextemplates.com)
%
% License:
% CC BY-NC-SA 3.0 (http://creativecommons.org/licenses/by-nc-sa/3.0/)
%
%%%%%%%%%%%%%%%%%%%%%%%%%%%%%%%%%%%%%%%%%

%----------------------------------------------------------------------------------------
%	PACKAGES AND OTHER DOCUMENT CONFIGURATIONS
%----------------------------------------------------------------------------------------

\documentclass[a4paper, 12pt]{article} % Font size (can be 10pt, 11pt or 12pt) and paper size (remove a4paper for US letter paper)

\usepackage[protrusion=true,expansion=true]{microtype} % Better typography
\usepackage{graphicx} % Required for including pictures
\usepackage[utf8]{inputenc}
\usepackage[margin=1.0in]{geometry}
\usepackage{url}
\usepackage{fancyhdr}
\usepackage{amsmath}
\usepackage{setspace}
\usepackage{enumitem}
\setlength\parindent{0pt} % Removes all indentation from paragraphs

\usepackage[T1]{fontenc} % Required for accented characters
\usepackage{times} % Use the Palatino font

\usepackage{listings}
\usepackage{color}
\lstset{mathescape}

\definecolor{dkgreen}{rgb}{0,0.6,0}
\definecolor{gray}{rgb}{0.5,0.5,0.5}
\definecolor{mauve}{rgb}{0.58,0,0.82}

\lstset{frame=tb,
   language=c++,
   aboveskip=3mm,
   belowskip=3mm,
   showstringspaces=false,
   columns=flexible,
   basicstyle={\small\ttfamily},
   numbers=none,
   numberstyle=\tiny\color{gray},
   keywordstyle=\color{blue},
   commentstyle=\color{dkgreen},
   stringstyle=\color{mauve},
   breaklines=true,
   breakatwhitespace=true
   tabsize=3
}
\linespread{1.00} % Change line spacing here, Palatino benefits from a slight increase by default

\makeatletter
\renewcommand{\@listI}{\itemsep=0pt} % Reduce the space between items in the itemize and enumerate environments and the bibliography

\renewcommand\abstractname{Résumé}
\renewcommand\refname{Références}
\renewcommand\contentsname{Table des matières}
\renewcommand{\maketitle}{ % Customize the title - do not edit title and author name here, see the TITLE block below
\begin{center} % Right align

\vspace*{25pt} % Some vertical space between the title and author name
{\LARGE\@title} % Increase the font size of the title

\vspace{125pt} % Some vertical space between the title and author name

{\large\@author} % Author name

\vspace{125pt} % Some vertical space between the author block and abstract
Dans le cadre du cours
\\INF8702 - Infographie avancée
\vspace{125pt} % Some vertical space between the author block and abstract
\\\@date % Date
\vspace{125pt} % Some vertical space between the author block and abstract

\end{center}
}

%----------------------------------------------------------------------------------------
%	TITLE
%----------------------------------------------------------------------------------------

\title{Rapport final: Génération de vague à l'aide de particule} 

\author{\textsc{Guillaume Arruda 1635805\\Raphael Lapierre 1644671} % Author
\vspace{10pt}
\\{\textit{École polytechnique de Montréal}}} % Institution

\date{8 Décembre 2015} % Date

%----------------------------------------------------------------------------------------

\begin{document}

\thispagestyle{empty}
\clearpage\maketitle % Print the title section
\pagebreak[4]
\tableofcontents
\pagebreak[4]
%----------------------------------------------------------------------------------------
%	En tête et pieds de page 
%----------------------------------------------------------------------------------------

\setlength{\headheight}{15.0pt}
\pagestyle{fancy}
\fancyhead[L]{INF8702}
\fancyhead[C]{}
\fancyhead[R]{Rapport final}
\fancyfoot[C]{\textbf{page \thepage}}

%----------------------------------------------------------------------------------------
%	ESSAY BODY
%----------------------------------------------------------------------------------------
\section{Introduction}
    \paragraph{}
    Dans le cadre du cours d'infographie avancée, un projet final au choix de l'étudiant 
    devait être réalisé. Lors des dernières semaines, ce dernier a donc été choisi et implémenté.
    Le but du présent travail est donc de présenter, sous forme de rapport, le fruit de notre
    travail des dernières semaines.

\subsection{Problématique}
    \paragraph{}
    De manière très générale, nous désirions, pour notre travail de session, implémenter 
    le rendu d'une surface d'eau de façon réaliste et de simuler des vagues sur celle-ci.
    Pour réussir à atteindre ce but, plusieurs éléments se devaient d'être présents pour notre
    simulation. Niveau lumière, il est impératif de pouvoir compter sur des effets de réflexion
    et de réfraction. En effet, une surface d'eau laissera pénétrer une partie de la lumière
    ce qui causera de la réfraction et une partie de la lumière sera réfléchie ce qui donnera
    la réflexion.

    \paragraph{}
    L'autre partie importante est d'avoir une simulation de vagues sur l'eau qui est assez
    proche de la réalité physique de la situation pour que l'illusion soit complète pour 
    l'humain. 

\section{Théorie}
\section{Implémentation}
\subsection{Introduction au moteur graphique}
Le moteur graphique a été écrit pour nous familiariser avec OpenGL 4.4. Il utilise les librairies suivantes:
\begin{description}
	\item[Assimp] pour le chargement des modèles 3D
	\item[DevIL] pour le chargement des textures
	\item[glfw3] pour la création des fenêtre
	\item[tinyxml2] pour la lecture des fichiers xml
	\item[glew] pour faciliter l'utilisation de OpenGL
\end{description}
\paragraph{}
Son fonctionnement est simple. Il utilise un arbre de rendu qui appelle la fonction "Update" qui met à jour les noeuds de l'arbre.
L'arbre de rendu appelle ensuite les fonctions "Render*" qui affiche les noeuds à l'écran. L'arbre de rendu est créé à l'aide du
fichier "MainRenderTree.xml" et l'environnement qui décrit les lumières est dans le fichier "MainEnvironnement.xml". Il est donc
simple de déplacer et de rajouter des éléments dans la scène.
\paragraph{}
La majorité des fonctionnalités implémentées pour ce projet se retrouve dans les fichiers du dossier "./src/Water/". Le noeud 
"Water" décrit dans le fichier "./src/RenderTree/Node/Water.h" représente la surface d'eau dans l'arbre de rendu.
\subsection{Génération et gestion des particules}
\subsection{Création de la carte de hauteurs}
\subsection{Réflection}
\subsection{Réfraction}
\section{Conclusion}
\section{Références}
\begin{figure}
	\centering
	\includegraphics[width=0.9\textwidth]{./PhotoRapport/EffetFinal.png}
	\caption{Effet final}
	\label{EffetFinal}
\end{figure}
\begin{figure}
	\centering
	\includegraphics[width=0.9\textwidth]{./PhotoRapport/NoSubdivide.png}
	\caption{Sans subdivision}
	\label{NoSubdivide}
\end{figure}
\begin{figure}
	\centering
	\includegraphics[width=0.9\textwidth]{./PhotoRapport/Reflection.png}
	\caption{Reflection}
	\label{Reflection}
\end{figure}
\begin{figure}
	\centering
	\includegraphics[width=0.9\textwidth]{./PhotoRapport/Refraction.png}
	\caption{Refraction}
	\label{Refractopm}
\end{figure}
\begin{figure}
	\centering
	\includegraphics[width=0.9\textwidth]{./PhotoRapport/WaterConservation.png}
	\caption{Conservation d'eau}
	\label{WaterConservation}
\end{figure}
%----------------------------------------------------------------------------------------
\end{document}
